\documentclass{article}

% set font encoding for PDFLaTeX or XeLaTeX
\usepackage{ifxetex}
\ifxetex
  \usepackage{fontspec}
\else
  \usepackage[T1]{fontenc}
  \usepackage[utf8]{inputenc}
  \usepackage{lmodern}
\fi

% used in maketitle
\title{Actividad 1}
\author{César Andrés Pérez Robinson \\
Departamento de Física \\
Universidad de Sonora}
\date{ 31 de Agosto de 2017}

% Enable SageTeX to run SageMath code right inside this LaTeX file.
% documentation: http://mirrors.ctan.org/macros/latex/contrib/sagetex/sagetexpackage.pdf
% \usepackage{sagetex}

\begin{document}
\maketitle
% son comentarios, no aparecen en el documento
\section{Introducción}
Se presenta un resumen de distintos comandos vistos en clase y
una descripción básica sobre ellos.

\subsection{Comandos de Bash}
\begin{verbatim}
Comando-Descripción
	Enlista los archivos y carpetas
	ej: ls (Enlista los archivos del directorio en uso).
	    ls -a (Enlista todo).
	    ls -al (Enlista todo en formato grande).
	    ls /dir (Enlista los archivos en el directorio 
        especificado /dir).
	    pwd (Nombra el directorio en uso).
	    cd (Cambia el directorio en uso).
	    rm (Borra los archivos).
	    mkdir (Crea un directorio).
	    mv (Mueve o renombra archivos y directorios).
	    man (Abre el manual).
	    cp (Copiar y pegar archivos y directorios).
	    * (Todos los nombres relacionados).
	    cat (Ver un archivo).
\end{verbatim}
\subsection{"Learning the Shell"}
\subsubsection{What is Shell?}
\begin{verbatim}
   	"The Shell" es un programa que toma comandos del teclado y 
    se los otorga al sistema operativo para ser ejecutados.

	 What's a "Terminal"?
	     Es un programa que abre una ventana y te deja 
         interactuar con "The Shell"
\end{verbatim}
\subsubsection{Navigation}
\begin{verbatim}
Introducción a los comandos: pwd (print working directory), 
cd (change directory) y ls (list files and directories).

a) File System Organization
   	Los archivos en Linux estan organizados en lo que se 
    llama una estructura de directorios jerárgicos.
    Significa que estan organizados en un patrón que 
    resembla un arbol de direcorios (folders), que pueden 
    contener archivos u otros directorios.

	 pwd
	     Para nombrar el directorio en uso
	 cd
	     Para cambiar de directorio en uso
	 ls
	     Para enlistar los contenidos del directorio en uso
	 ls /bin
	     Para enlistar los archivos del directorio /bin (o
	     cualquier directorio especificado)

b) Long Format
   	En el uso de ls -l
	     Nombre del Archivo
	     Tiempo de Modificación
     Tamaño
	     Grupo
	     Dueño
	     Permisos del Archivo
\end{verbatim}
\subsubsection{A Guided Tour}
\begin{verbatim}
Se encuentran una lista de directorios interesantes dignos a
explorar.

      /boot
      /etc
     	  Contiene configuración del sistema.
      /bin, /usr/sbin
     	  Contiene programas para el sistema de administración.
      /var
     	  Contien archivos que cambian mientras el sistema esta
     operacional.
      /lib
     	  Se guardan las librerias compartidas.
\end{verbatim}
\subsubsection{Manipulating Files}
\begin{verbatim}
Introducción a los comandos
		cp
		mv
		rm
		mkdir

a) Wildcards
	 * (Matches any characters)
	 ? (Mathces any single character)
Usando Wildcards es posible construir una selección
sofisticada de criterios para archivos.
	 *        Todos los nombres de archivos
	g*        Todos los archivos que empiecen con la letra g
	b*.txt    Todos los archivos que empiecen con la letra b 
          y terminen en .txt
	Data???   Cualquier archivo que empiece con la palabra 
          Data y exactamente tres más caracteres.
	[abc]*    Cualquier archivo que empiece con la letra
	          a, b o c, seguido de cualquier otros caracteres.

b) cp (Este programa copia archivos y directorios)
	Comandos
	cp file1 file2    Copia los contenidos del file1 al file2.
                  Si  file2 no existe, es creado, de la otra 
                  manera file2 es sobreescrito con los 
                  contenidos de file1.
	cp i file1 file2  Como el comando anterior, sin embargo 
                  si el  file2 existe, el usuario es solicidato 
                  antes de ser sobreescrito con file1.
	cp file1 dir1     Copia los contenidos del file1 dentro del 
                  directorio dir1.
	cp -R dir1 dir2   Copia los contenidos del directorio dir1.
                  Si el directorio dir2 no existe, es creado. 
                  De lo contrario, crea un directorio llamado 
                  dir1 dentro de otro llamado dir2.

c) mv (Mueve o renombra archivos y diccionarios dependiendo
	de como es usado)
	   	Comandos
	mv file1 file2             Si file2 no existe, entonces file1
                           es renombrado file2. Si file2 existe,
                           sus contenidos son remplacados con el 
                           contenido de file1.
	mv -i file1 file2          Como en el comando anterior,
	                           solo que ahora el usuario  es 
                           solicitado antes de sobreescribir 
                           con los contenidos de file2.
	mv file1 file2 file3 dir1  Los archivos son movidos al 
                           directorio dir1.
    mv dir1 dir2           Si dir2 no existe entonces dir1 es 
                           renombrado dir2. Si dir2 existe, el 
                           directorio dir1 es movido al dir2.

d) rm (Borra archivos y directorios)
	     Comandos
	rm file1 file2      Borra los archivos file1 file2
	rm -i file1 file2   Como el comando anterior pero el usuario
                    es solicitado antes de borrar cada archivo.
	rm -r dir1 dir2     Borra los directorios dir1 y dir2 
                    junto con sus archivos.
	CUIDADO CON rm !!!

e) mkdir (Crea directorios)

f) Usando Comandos con Wildcards
	cp *.txt text_files             Copia los archivos del 
                                directorio en uso con nombre al 
                                directorio existente llamado 
                                "text_files".
	mv my_dir ../*.bak my_new_dir   Mueve el subdirectorio my_dir
                                y los archivos cuya terminación
                                sea "bak" del directorio padre del
                                directorio en uso al directorio
                                existente llamado my_new_dir.
	rm *~                           Borra todos los archivos del 
                                directorio en uso cuya terminación
                                sea "~"

\end{verbatim}
\subsubsection{Pruebas}
\begin{equation}
F=ma=a \frac{dv}{dt}
\end{equation}
\end{document}
