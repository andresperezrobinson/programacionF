\documentclass{article}

% set font encoding for PDFLaTeX or XeLaTeX
\usepackage{ifxetex}
\ifxetex
  \usepackage{fontspec}
\else
  \usepackage[T1]{fontenc}
  \usepackage[utf8]{inputenc}
  \usepackage{lmodern}
  \usepackage{graphicx}
\fi

% used in maketitle
\title{Tiro Parabólico con Resistencia del Aire}
\author{César Andrés Pérez Robinson}

% Enable SageTeX to run SageMath code right inside this LaTeX file.
% documentation: http://mirrors.ctan.org/macros/latex/contrib/sagetex/sagetexpackage.pdf
% \usepackage{sagetex}

\begin{document}
\maketitle
\section{Descripción de la Actividad}
Para esta actividad se genera una gráfica con gnuplot sobre el tiro parabólico de un objeto con resistencia del aire. Guiándonos de ecuaciones de movimiento, se resuelven utilizando el método de integración de Euler. Se calcula la velocidad terminal, por lo que es necesario conocer la geometría del objeto, puesto que de ella depende el coeficiente de arrastre.

\section{Geometría del objeto}
Se escogió un objeto similar a una pelota de beisból cuyas dimensiones son:

Radio: 0.05 m
 
Masa: 0.250 kg
 
El coeficiente de arrastre de la esfera: 0.47
\section{Gráfica y Código Utilizado}

La grafica de Figura 1 fue generada a través de gnuplot y muestra el movimiento del proyectil con velocidades iniciales 10, 20, 30, 40, 50, 60, 70, 80, 90 y 100.


\begin{figure}
    \includegraphics[width=\linewidth]{Tiro.png}
    \caption{Resistencia del Aire}
    \label{fig:Tiro}
\end{figure}


El código utilizado es el siguiente:
\begin{verbatim}
program vector

  implicit none


  ! definimos constantes
  real, parameter :: g = 9.81
  real, parameter :: pi = 3.1415927
  real, parameter :: cd = 0.47
  real, parameter :: de = 1.22
  real, parameter :: r = 0.05
  real, parameter :: m = 0.250
  real, parameter :: dt = 0.01
 

  ! definimos las variables
  real :: u, t, a
  integer :: i, j
  integer, parameter :: ntimes = 1000
  real, dimension (1 : ntimes) :: x, y, vx, vy
  real :: fi, fj 
  real, parameter :: Area = pi * r * r
  real, parameter :: Vet = SQRT(2 * m * g / (de * Area * cd))
  real, parameter :: c = m * g / Vet

  open(unit=15, file="salida.dat", status="unknown")
  
  ! Leer los valores para el ángulo a y la velicidad inicial u desde
  ! la terminal
  write (*,*) 'Dame el ángulo'
  read (*,*) a 

  
  ! convirtiendo ángulo a radianes
  a = a * pi / 180.0
  
  ! las ecuaciones de la posición en x y y
 ! x = u * cos(a) * i
  !y = u * sin(a) * i - 0.5 * g * i * i

do j = 10, 100, 10
   fj = float(j)
   u = fj
   
  write(15,*) " "

  do i = 1, 2
     fi = float(i)
     t = fi * dt
  
  x(i) = u * cos(a) * t
  y(i) = u * sin(a) * t - 0.5 * g * t * t
  vx(i) = u
  vy(i) = u * sin(a) - g * t

  write(15,*) x(i), y(i)
  
end do

  
  do i = 3, ntimes
     fi = float(i)
     t = fi * dt
 
      
     vx(i) = vx(i - 1) * (1 - dt * c / m)
     
     x(i) = x(i - 1) + dt * vx(i - 2) * (1 - dt * c / m)
     
     vy(i) = vy(i - 1) * (1 - dt * c / m) - dt * g
     
     y(i) = y(i- 1) + dt * vy(i - 2) - dt * dt * g + dt * c / m *&
          & vy(i - 2)

  
  if (y(i) < 0) exit
  
  write(15,*) x(i), y(i)
  

end do
end do

 close(unit=15)
 

  !write(*,*) 'Modulus squared = ',x

end program vector

\end{verbatim}

\end{document}
