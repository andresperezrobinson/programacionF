\documentclass{article}

% set font encoding for PDFLaTeX or XeLaTeX
\usepackage{ifxetex}
\ifxetex
  \usepackage{fontspec}
\else
  \usepackage[T1]{fontenc}
  \usepackage[utf8]{inputenc}
  \usepackage{lmodern}
\fi
\usepackage{graphicx}

% used in maketitle
\title{Órbita de la Luna}
\author{César Andrés Pérez Robinson}

% Enable SageTeX to run SageMath code right inside this LaTeX file.
% documentation: http://mirrors.ctan.org/macros/latex/contrib/sagetex/sagetexpackage.pdf
% \usepackage{sagetex}

\begin{document}
\maketitle
\section{Movimiento de la Luna}
En este trabajo se agrega el movimiento de la Luna al sistema Tierra-Sol.
Se supone que la Tierra gira alrededor del Sol en un movimiento circular. De la misma manera, supondremos que la Luna gira alrededor de la Tierra siguiento una órbita circular y que tarda 27 días, 7 horas, 43 minutos y 11.5 segundos en completar una vuelta a la tierra.

\section{Código Utilizado}
A continuación se muestra el código FORTRAN utilizado para obtener los datos correspondientes. Se utiliza una subrutina.
\begin{verbatim}

subroutine orbitas (Rad1, Rad2, Ang1, Ang2, tiempo, xpos, ypos)
  double precision, intent(in) :: Rad1, Rad2, Ang1, Ang2, tiempo
  double precision, intent(out) :: xpos, ypos

  xpos = Rad1 * dcos(Ang1) + Rad2 * dcos(Ang2)

  ypos = Rad1 * dsin(Ang1) + Rad2 * dsin(Ang2)

end subroutine orbitas

  program orbita

    implicit none
  double precision :: T = 365.26d0, omega, dteta, omegal 
  double precision :: G = 4.302d-3, pi
  double precision :: R = 1.5d+8
  double precision :: Ms = 1.98855d+30
  double precision :: Vt, ds, dt, P, a, fi, al
  integer :: i
  integer, parameter :: ntimes = 360
  double precision, dimension (0 : ntimes) :: x, y
  double precision :: Rad1 = 1.5d+8
  double precision :: Rad2
  double precision :: Ang1
  double precision :: Ang2
  double precision :: tl = 27.32166088d0
  double precision :: xpos, ypos  
  
  parameter (pi = 3.141593d0)

  open(unit=5, file="Moon.dat", status="unknown")
  
  Rad2 = Rad1 / 5.0d0
  
  !Cambiando angulos a radianes
  
  !a = a * pi / 180.0d0

  !Velocidad angular

  omega = 2.0d0 * pi / T

  omegal = 2.0d0 * pi / tl

  !Velocidad Tangente

  Vt = R * omega

  !Definiendo ds y dt
  
  dt = 365.260d0 / 360.0d0

  ds = Vt * dt

  !Para obtener el delta de teta

  dteta = 2.0d0 * pi / 3.6d2 

  !Para las posiciones

  do i = 0, ntimes
     fi = dble(i)
     
     Ang2 = fi * omegal

     Ang1 = fi * omega

     call orbitas(Rad1, Rad2, Ang1, Ang2, 5.0d0, xpos, ypos)
     x(i) = xpos
     y(i) = ypos

     write(5,*) x(i), y(i)

  end do

  close(unit=5)
  
  end program

\end{verbatim}

\section{Gráfica de los datos}
La Figura 1 muestra los datos obtenidos con el código anterior. Adicionalmente, se le agrega a la gráfica los datos de la órbita de la tierra. Por lo que se obtiene la órbita de la Luna y la órbita de la tierra en el sistema Tierra-Sol.

\begin{figure}
    \includegraphics[width=\linewidth]{GraficaTL.png}
    \caption{Órbita Luna y Tierra alrededor del Sol}
    \end{figure}

\end{document}
