\documentclass{article}

% set font encoding for PDFLaTeX or XeLaTeX
\usepackage{ifxetex}
\ifxetex
  \usepackage{fontspec}
\else
  \usepackage[T1]{fontenc}
  \usepackage[utf8]{inputenc}
  \usepackage{lmodern}
\fi
\usepackage{graphicx}

% used in maketitle
\title{Órbita de la Tierra}
\author{César Andrés Pérez Robinson}

% Enable SageTeX to run SageMath code right inside this LaTeX file.
% documentation: http://mirrors.ctan.org/macros/latex/contrib/sagetex/sagetexpackage.pdf
% \usepackage{sagetex}

\begin{document}
\maketitle

\section{Órbita}
Se presenta un código y una gráfica que simula el movimiento de la Tierra alrededor del Sol.

Suponiendo que la Tierra giera alrededordel Sol siguiendo un movimiento circular. Por lo tanto, se fija la distancia en la que la Tierra se mueve alrededor del Sol una distancia R fija (150,000,000 km) y tarda 365.26 días en dar una vuelta completa.

\subsection{Código utilizado}
\begin{verbatim}

  program orbita

    implicit none
  double precision :: T = 365.26d0, omega, dteta 
  double precision :: G = 4.302d-3, pi
  double precision :: R = 1.5d+8
  double precision :: Ms = 1.98855d+30
  double precision :: Vt, ds, dt, P, a, fi
  integer :: i
  integer, parameter :: ntimes = 360
  double precision, dimension (0 : ntimes) :: x, y
  
  parameter (pi = 3.141593d0)

  open(unit=3, file="tierra.dat", status="unknown")
  

  !Cambiando angulos a radianes
  
  a = a * pi / 180.0d0

  !Velocidad angular

  omega = 2.0d0 * pi / T

  !Velocidad Tangente

  Vt = R * omega

  !Definiendo ds y dt
  
  dt = 365.260d0 / 360.0d0

  ds = Vt * dt

  !Para obtener el delta de teta

  dteta = 2.0d0 * pi / 3.6d2

  !Para las posiciones

  do i = 0, ntimes
     fi = float(i)

     x(i) = R * dcos(fi * dteta)
     y(i) = R * dsin(fi * dteta)

     write(3,*) x(i), y(i)

  end do
  

  

  close(unit=3)

  
  end program
\end{verbatim}

\subsection{Gráfica resultante}
La siguiente gráfica Figura 1 fue desarrollada en gnuplot con los datos obtenidos del programa anterior.
\begin{figure}
    \includegraphics[width=\linewidth]{grafica.png}
    \caption{Movimiento circular de la Tierra}
    \label{fig:Movimiento circular de la Tierra}
\end{figure}
\end{document}
