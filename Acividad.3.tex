\documentclass{article}

% set font encoding for PDFLaTeX or XeLaTeX
\usepackage{ifxetex}
\ifxetex
  \usepackage{fontspec}
\else
  \usepackage[T1]{fontenc}
  \usepackage[utf8]{inputenc}
  \usepackage{lmodern}
  \usepackage{graphicx}
\fi

% used in maketitle
\title{Gráfica de Tiros Parabólicos}
\author{César Andrés Pérez Robinson}

% Enable SageTeX to run SageMath code right inside this LaTeX file.
% documentation: http://mirrors.ctan.org/macros/latex/contrib/sagetex/sagetexpackage.pdf
% \usepackage{sagetex}

\begin{document}
\maketitle

\section{Movimiento Parabólico}

Se denomina movimiento parabólico, al movimiento relizado por cualquier parábola. Se corresponde con la trayectoria ideal de un proyectil que se mueve en un medio que no ofrece resistencia al avance y que está sujeto a un campo gravitatorio uniforme. El movimiento parabólico es un ejemplo deun movimiento realizado porun objeto en dos dimensiones o sobre un plano.

\subsection{Distintos Tiros con Misma Velocidad Inicial y Distintos Ángulos}

Se utiliza el graficador Gnuplot para realizar la Figura 1, que muestra distintos tiros, todos realizados con velocidad inicial 10 m/, pero con los distintos ángulos iniciales: 15º, 30º, 45º, 60º, 75º y 90.

\begin{figure}

\includegraphics[width=\linewidth]{Grafica}

\caption{Distintos Tiros}

\label{fig:Distintos Tiros1}

\end{figure}

\subsection{Códigos Utilizados}

Para generar la gráfica de la Figura 1, se utilizó el siguiente código:

\begin{verbatim}
program vector

  implicit none


  ! definimos constantes
  real, parameter :: g = 9.81
  real, parameter :: pi = 3.1415927
  real, parameter :: deltat = .01

  ! definimos las variables
  real :: u, a, t
  real, dimension (1000) :: x, y
  integer :: i, j
  integer, parameter :: ntimes = 1000
  real :: fi, fj 
 

  ! Leer los valores para el ángulo a y la velicidad inicial u desde
  ! la terminal
  write (*,*) 'Dame la velocidad inicial y el ángulo'
  read (*,*) a, u

  
  ! convirtiendo ángulo a radianes
  a = a * pi / 180.0
  
  ! las ecuaciones de la posición en x y y
 ! x = u * cos(a) * i
  !y = u * sin(a) * i - 0.5 * g * i * i
  
  do j = 15, 90, 15
     fj = float(j)
     a = fj * pi / 180.0
  write(*,*) " "
  do i = 1, ntimes
     fi = float(i)
     t = fi * deltat
  x(i) = u * cos(a) * t
  y(i) = u * sin(a) * t - 0.5 * g * t * t
  if (y(i) < 0) exit
  write(*,*) x(i), y(i)
  

end do
end do

 
 

  !write(*,*) 'Modulus squared = ',x

end program vector

\end{verbatim}



\end{document}
