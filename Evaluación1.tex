\documentclass{article}

% set font encoding for PDFLaTeX or XeLaTeX
\usepackage{ifxetex}
\ifxetex
  \usepackage{fontspec}
\else
  \usepackage[T1]{fontenc}
  \usepackage[utf8]{inputenc}
  \usepackage{lmodern}
\fi

% used in maketitle
\title{Evaluación 1}
\author{César Andrés Pérez Robinson}

% Enable SageTeX to run SageMath code right inside this LaTeX file.
% documentation: http://mirrors.ctan.org/macros/latex/contrib/sagetex/sagetexpackage.pdf
% \usepackage{sagetex}

\begin{document}
\maketitle
\section{Volumen y área de una esfera}

\subsection{Código utilizado}
\begin{verbatim}


!! cylinder.f90
!! 
!! Made by (Andres Perez Robinson)
!! Login   <andres@ltsp139.example.com>
!! 
!! Started on  Mon Oct 30 11:10:23 2017 Andres Perez Robinson
!! Last update Time-stamp: <2017-oct-30.lunes 11:38:19 (andres)>
!

  program cylinder

! Calculate the surface area of a cylinder.
!
! Declare variables and constants.
! constants=pi
! variables=radius squared and height

  implicit none    ! Require all variables to be explicitly declared

  integer :: ierr
  character(1) :: yn
  real :: radius, height, area, volumen
  real, parameter :: pi = 3.141592653589793

  interactive_loop: do

!   Prompt the user for radius and height
!   and read them.

    write (*,*) 'Enter radius'
    read (*,*,iostat=ierr) radius

!   If radius and height could not be read from input,
!   then cycle through the loop.

    if (ierr /= 0) then
      write(*,*) 'Error, invalid input.'
      cycle interactive_loop
    end if

!   Compute area.  The ** means "raise to a power."

    volumen = pi * radius**3 * 4 / 3

    area = 4 * pi * radius**2

!   Write the input variables (radius, height)
!   and output (area) to the screen.

    write (*,'(1x,a7,f9.2,5x,a7,f9.2,5x,a5,f9.2)') &
      'volumen=',volumen,'area=',area

    yn = ' '
    yn_loop: do
      write(*,*) 'Perform another calculation? y[n]'
      read(*,'(a1)') yn
      if (yn=='y' .or. yn=='Y') exit yn_loop
      if (yn=='n' .or. yn=='N' .or. yn==' ') exit interactive_loop
    end do yn_loop

  end do interactive_loop

end program cylinder

\end{verbatim}
\subsection{Datos}
A continuación se presentan una serie de radiosdistintos con sus respectivos volúmenes y áreas.
\begin{table}[]
\centering
\caption{My caption}
\label{my-label}
\begin{tabular}{lll}
\textbf{Radio} & \textbf{Volumen} & \textbf{Área} \\
1              & 4.19             & 12.57         \\
2              & 33.51            & 50.27         \\
3              & 113.10           & 113.10        \\
//             & //               & //            \\
15             & 14137.17         & 2827.43       \\
20             & 33350.32         & 5026.55       \\
25             & 65449.85         & 7853.98      
\end{tabular}
\end{table}


\section{Medias aritmética y armónica}

\subsection{Código utilizado}
\begin{verbatim}

program summation
implicit none
integer ::  a
real :: aritmetica, armonica
real :: n, sum, sumi
print*, "This program performs summations. Enter 0 to stop."
open(unit=10, file="SumData.DAT")

sum = 0
n = 0
sumi = 0.0
do
 print*, "Add:"
 read*, a
 if (a == 0) then
  exit
 else
sum = sum + a
end if

   
   n = (n + 1)
   sumi = sumi + 1.0 / a
   aritmetica = sum / n
   armonica = n / sumi


 write(10,*) a
end do

print*, "armonica =", armonica
write(10, *) "armonica =", armonica

print*, "aritmetica =", aritmetica
write(10, *) "aritmetica =", aritmetica

print*, "Summation =", sum
write(10,*) "Summation =", sum
close(10)

end

\end{verbatim}
\subsection{Algunos ejemplos}
Para los datos 
\begin{verbatim}
           1
           2
           3
           4
           5
           6
           7
           8
           9
          10
 Media armónica =   3.41417122    
 Media aritmética =   5.50000000    
 Sumatoria =   55.0000000    
\end{verbatim}
Para los datos
\begin{verbatim}
           1
           4
           4
 Media armónica =   2.00000000    
 Media aritmética =   3.00000000    
 Sumatoria =   9.00000000 
 \end{verbatim}
 
\end{document}
