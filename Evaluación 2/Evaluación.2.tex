\documentclass{article}

% set font encoding for PDFLaTeX or XeLaTeX
\usepackage{ifxetex}
\ifxetex
  \usepackage{fontspec}
\else
  \usepackage[T1]{fontenc}
  \usepackage[utf8]{inputenc}
  \usepackage{lmodern}
\fi

% used in maketitle
\title{Evaluación 2}
\author{César Andrés Pérez Robinson}

% Enable SageTeX to run SageMath code right inside this LaTeX file.
% documentation: http://mirrors.ctan.org/macros/latex/contrib/sagetex/sagetexpackage.pdf
% \usepackage{sagetex}

\begin{document}
\maketitle
\section{Serie de MacLaurin}
La serie de Taylor es una aproximación de funciones mediante una serie de potencias o suma de potencias enteras de polinomios llamados términos de la serie, dicha suma se calcula a partir de las derivadas de la función para un determinado valor. Si esta serie está centrada sobre el punto cero, se le denomina serie de McLaurin.

\section{Código}
El siguiente código usa una funciona para la Serie de McLaurin para aproximar la función exponencial en el punto x=1, utilizando n=20 términos.


\begin{verbatim}
! ----------- Begin ------------

!taylor.f90

program taylor

    implicit none                  
real (kind=8) :: x, exp_true, y
    real (kind=8), external :: exptaylor
    integer :: n

    n = 20               ! number of terms to use
    x = 1.0
    exp_true = exp(x)
    y = exptaylor(x,n)   ! uses function below
    print *, "x = ",x
    print *, "exp_true  = ",exp_true
    print *, "exptaylor = ",y
    print *, "error     = ",y - exp_true

end program taylor

!==========================
function exptaylor(x,n)
!==========================
    implicit none

    ! function arguments:
    real (kind=8), intent(in) :: x
    integer, intent(in) :: n
    real (kind=8) :: exptaylor

    ! local variables:
    real (kind=8) :: term, partial_sum
    integer :: j

    term = 1.
    partial_sum = term

    do j=1,n
        ! j'th term is  x**j / j!  which is the 
        previous term times x/j:
        term = term*x/j   
        ! add this term to the partial sum:
        partial_sum = partial_sum + term   
        enddo
     exptaylor = partial_sum  ! this is the value returned
end function exptaylor

! --------  End -------------
\end{verbatim}

\subsection{Resultados}
\begin{verbatim}
x =    1.0000000000000000     
 exp_true  =    2.7182818284590451     
 exptaylor =    2.7182818284590455     
 error     =    4.4408920985006262E-016
\end{verbatim}

\subsection{Intento de código}
\begin{verbatim}

subroutine e(x, y, npts, sum)
  implicit none 
  real (kind=8), dimension(100), intent(in) :: x
  real (kind=8), dimension(100), intent(out) :: y
  integer (kind=8), intent(in) :: npts
  integer (kind=8) , intent(out) :: sum

  !variables locales
  real(kind=8) :: term, partial_sum
  integer :: j

  term = 1.0
  partial_sum = term

  do j = 1, sum
     term = term * x(j) / float(j)
     partial_sum = partial_sum + term
     y(j) = partial_sum
  end do

  

  end subroutine e

program sub

  implicit none
  real (kind=8) :: exp_true, fj, fi
  integer (kind=8) :: sum, npts
  integer :: j, i 
  real (kind=8), dimension(100) :: x
  real (kind=8), dimension(100) :: y
  integer, dimension(15) :: n
  real, dimension(100,15) :: sum

  open(unit=2, file="exponencial.dat", status="unknown")

  npts = 100
  
  do j= 1, 15, 2
     n(j) = j
  end do
  

  do i = 0, npts
     fi = float(i)
     x(i) = fi /10.0
  end do
  
     call e(x, y, npts, sum)
     exp_true = exp(x(i))
     print *, "x = ", x
     print *, "exp_true = ",exp_true
     print *, "exptaylor = ",y
     print *, "error  = ",y - exp_true

     do i = 1, npts
        
        write(2,*) x(i), y(i)
        
  end do

  write(2,*) ' ' 

 
     
  

  close(unit=2)
  
end program sub

  

\end{verbatim}
 El código anterior fue el avance realizado para la actividad dos. Esta no fue completada.

\end{document}
